\documentclass[24pt,handout]{beamer}

\usepackage{polski}
\usepackage[utf8]{inputenc}

\usepackage{listings}
\usepackage{color}

\title[ReactiveX]{ReactiveX - okiełznać asynchroniczność}
\date{Gliwice 2.X.2015}
\author{Mateusz Łyczek}

\usetheme{FDD2015}

\lstdefinestyle{mystyle}{
    basicstyle=\footnotesize
}
\lstset{style=mystyle}

\tikzset{event/.style={circle,draw=FDDOrange,fill=FDDOrange!40,thick,inner sep=0pt,minimum size=6mm},
         error/.style={draw=red,line width=2pt},
         termination/.style={line width=2pt}}

%\setbeameroption{show notes}

\begin{document}
\begin{frame}[plain]
    \titlepage
\end{frame}

\section{Wstęp}

\begin{frame}
    \begin{columns}
        \begin{column}{.5\textwidth}
            \begin{center}
                \includegraphics[height=.6\textheight]{images/multitasking.jpg}
            \end{center}
        \end{column}
        \begin{column}{.5\textwidth}
            \begin{center}
                \includegraphics<2>[width=.6\textwidth]{images/chip.pdf}
            \end{center}
        \end{column}
    \end{columns}

    \note<2>[item]{Dzisiejsze aplikacje są asynchroniczne}
    \note<2>[item]{Pobieramy dane z~wielu źródeł}
    \note<2>[item]{Reagujemy na push message z~serwera}
    \note<2>[item]{Reagujemy na działanie użytkownika}
    \note<2>[item]{Mamy wielordzeniowe komputery}
    \note<2>[item]{Programy coraz bardziej współbieżne}
\end{frame}

\begin{frame}
    \frametitle{Problemy z~asynchronicznością}

    \begin{columns}
        \begin{column}{.5\textwidth}
            \begin{itemize}
                \item<1-> Różne API w~każdym języku
                \begin{itemize}
                    \item<1-> \texttt{Thread} w~Javie
                    \item<1-> \texttt{async/await} w~.NET
                    \item<1-> callbacks, \texttt{Promise} w~JavaScript
                    \item<1-> \dots
                \end{itemize}
                \item<2-> Ciężka obsługa błędów (\texttt{try/catch}?)
                \item<3-> Callback hell
                \item<4-> Zdarzenia?
                \item<5-> Kompozycja?
            \end{itemize}
        \end{column}

        \begin{column}{.5\textwidth}
            \includegraphics<3->[width=\textwidth]{images/callback_hell.jpg}
        \end{column}
    \end{columns}
\end{frame}

\section{ReactiveX}

\begin{frame}
    \begin{center}
        {\Huge \raisebox{-.2\height}{\includegraphics[height=0.1\paperheight]{images/Rx_Logo_M.png}} ReactiveX}

        \vspace{20pt}

        {\Large An API for asynchronous programming with~\alert<2>{observable streams}}

        \vspace{20pt}

        ReactiveX is a combination of the best ideas from the Observer pattern, the Iterator pattern and functional programming
    \end{center}

    \note<2>{Chcę opowiedzieć o~ReactiveX i~o~tym jak może nam z~tymi problemami pomóc.

    Twórcy opisują ReactiveX jako API do asynchronicznego programowania przy użyciu obserwowalnych strumieni. I~to właśnie te obserwowalne
    strumienie są centrum ReactiveX.}
\end{frame}

\subsection{Observable streams}

\begin{frame}[fragile]
    \frametitle{Observable streams}

    \begin{center}
        \begin{tikzpicture}
            \draw[->,line width=2pt] (0,0) -- (10,0);

            \node[event] at (1,0) {A};
            \node[event] at (2,0) {B};
            \node[event] at (5,0) {C};

            \draw[error] (6.8,0.2) -- (7.2,-0.2);
            \draw[error] (7.2,0.2) -- (6.8,-0.2);

            \draw[termination] (9,0.2) -- (9,-0.2);
        \end{tikzpicture}
    \end{center}

    \vspace{20pt}

    \begin{center}
        \def\arraystretch{1.8}
        \begin{tabular}{cl}
            \tikz[baseline=-0.7ex]{\node[event] at (0,0){}} & element strumienia \\

            \tikz[x=1.2ex,y=1.2ex,baseline=-0.7ex]{\draw[error] (0,1) -- (1.6,-1) (0,-1) -- (1.6,1);} & błąd \\

            \tikz[x=1.2ex,y=1.2ex,baseline=-0.7ex]{\draw[termination] (0,1) -- (0,-1)} & koniec strumienia
        \end{tabular}
    \end{center}

    \note{No dobra, ale to przecież nic nowego! Mamy event busy, czy zwykłe eventy użytkownika (klikanie myszki, itp.),
    na które możemy nasłuchiwać i~reagować na nie.}
\end{frame}

\begin{frame}
    \frametitle{Co może być elementem strumienia?}

    \pause

    \begin{center}
        {\Huge wszystko}
    \end{center}
\end{frame}

\begin{frame}
    \frametitle{Co może być elementem strumienia?}

    \begin{itemize}
        \item zdarzenia użytkownika (np. klikanie myszki)
        \item \texttt{String}
        \item odpowiedź serwera na zapytanie HTTP
        \item tweet
        \item pojedynczy wiersz wyniku zapytania SQL
        \item \dots
    \end{itemize}

    \note{No ale obserwowanie takiego strumienia to przecież zwykły wzorzec obserwatora!}
\end{frame}

\subsection{Stream operators}

\begin{frame}
    \frametitle{Łączenie strumieni}

    \begin{center}
        \begin{tikzpicture}
            \draw[->,line width=2pt] (0,5) -- (10,5);

            \node[event] at (1,5) {A};
            \node[event] at (4,5) {B};
            \node[event] at (7,5) {C};
            \node[event] at (8,5) {D};

            \draw[->,line width=2pt] (0,3) -- (10,3);

            \node[event] at (2,3) {1};
            \node[event] at (3,3) {2};
            \node[event] at (5,3) {3};
            \node[event] at (6,3) {4};

            \draw (0,1) -- (10,1) -- (10,2) -- (0,2) -- (0,1);
            \node at (5,1.5) {ZIP};

            \draw[->,line width=2pt] (0,0) -- (10,0);

            \node[event] at (2,0) {1A};
            \node[event] at (4,0) {2B};
            \node[event] at (7,0) {3C};
            \node[event] at (8,0) {4D};
        \end{tikzpicture}
    \end{center}

    \note{Tutaj tkwi dalsza część siły ReactiveX.

    Observable (strumienie) są w~ReactiveX first classc itizen, bo można je łączyć, tworzyć nowe, przekazywać jako parametry itp.}
\end{frame}

\begin{frame}
    \frametitle{Operatory}

    \begin{center}
        \def\arraystretch{1.8}
        \begin{tabular}{cccc}
            \texttt{From} &~\texttt{Join} &~\texttt{Range} & \texttt{Interval} \\
            \texttt{Just} &~\texttt{FlatMap} &~\texttt{GroupBy} & \texttt{Map} \\
            \texttt{Scan} &~\texttt{Window} &~\texttt{Debounce} & \texttt{Distinct} \\
            \texttt{ElementAt} &~\texttt{Filter} &~\texttt{First} & \texttt{Last} \\
            \texttt{Sample} &~\texttt{Skip} &~\texttt{SkipLast} & \texttt{Take} \\
            \texttt{TakeLast} &~\texttt{And} &~\texttt{CombineLatest} & \texttt{Join} \\
            \texttt{Merge} &~\texttt{Switch} &~\texttt{Zip} & \texttt{Retry}
        \end{tabular}

        \vspace{2ex}

        \dots
    \end{center}

    \note{Każda implementacja ReactiveX w~konkretnym języku programownia ma jeszcze swoje operatory specyficzne dla danego języka}
\end{frame}

\subsection{Summary}

\begin{frame}
    \frametitle{Tylko Java i~JavaScript?}

    \begin{columns}
        \begin{column}{.5\textwidth}
            \begin{itemize}
                \item Java
                \item JavaScript
                \item C\#
                \item Scala
                \item Clojure
                \item C++
            \end{itemize}
        \end{column}
        \begin{column}{.5\textwidth}
            \begin{itemize}
                \item Ruby
                \item Python
                \item Groovy
                \item JRuby
                \item Kotlin
                \item Swift
            \end{itemize}
        \end{column}
    \end{columns}
\end{frame}

\begin{frame}
    \frametitle{Co dalej?}

    \begin{block}{\url{http://reactivex.io}}
        Główne źródło wiedzy na temat ReactiveX wraz z~linkami do dokumentacji dla wszystkich implementacji w~różnych językach
    \end{block}

    \begin{block}{\url{https://gist.github.com/staltz/868e7e9bc2a7b8c1f754}}
        ``The introduction to Reactive Programming you've been missing''
    \end{block}

    \begin{block}{\url{https://github.com/mlyczek/fdd2015-reactivex}}
        Slajdy z~tej prezentacji oraz ich kod (\LaTeX)
    \end{block}

\end{frame}

\begin{frame}
    \frametitle{Co dalej?}
    \includegraphics[width=\textwidth]{images/coding.jpg}
\end{frame}

\begin{frame}
    \begin{center}
        \Huge ?
    \end{center}
\end{frame}

\endpage
\end{document}
